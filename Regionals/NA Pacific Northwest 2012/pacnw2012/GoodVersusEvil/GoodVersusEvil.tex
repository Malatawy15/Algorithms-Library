\documentclass{article}

\usepackage{geometry}
\usepackage{verbatim}
\usepackage{tabularx}

\title{Problem A: Good versus Evil, Who Will Win?  }
\date{}

\begin{document}
\maketitle

Middle Earth is about to go to war.  The forces of good will have many
battles with the forces of evil.  Different races will certainly be involved.
Each race has a certain `worth' when battling against others.  
On the side of good we have the following races, with their associated
worth:
\begin{quote}
Hobbits - 1 \\
Men - 2 \\
Elves - 3 \\
Dwarves - 3 \\
Eagles - 4 \\
Wizards - 10 
\end{quote}
On the side of evil we have:
\begin{quote}
Orcs - 1 \\
Men - 2 \\
Wargs - 2 \\
Goblins - 2 \\
Uruk Hai - 3 \\
Trolls - 5 \\
Wizards - 10
\end{quote}

Although weather, location, supplies and valor play a part in any battle,
if you add up the worth of the side of good
and compare it with the worth of the side of evil, the side with the 
larger worth will tend to win.

Thus, given the count of each of the races on the side of good, 
followed by the count of each of the races on the side of evil,
determine which side wins.  

\section{Input}

The first line of input will contain an integer greater than 0 signifying
the number of battles to process.  Information for each battle will consist
of two lines of data as follows.

First, there will be a line containing the count of each race on
the side of good.  Each entry will be separated by a single space.  The
values will be ordered as follows: Hobbits, Men, Elves, Dwarves, Eagles,
Wizards.

The next line will contain the count of each race on the side of evil
in the following order: Orcs, Men, Wargs, Goblins, Uruk Hai, Trolls, Wizards.

All values are non-negative integers.  The resulting sum of the worth for each
side will not exceed the limit of a 32-bit integer.

\section{Output}

For each battle, print ``\verb+Battle+'' followed by a single space, followed by 
the battle number starting at 1, followed by a ``\verb+:+'', followed by a single
space.  Then print ``\verb+Good triumphs over Evil+'' if good wins.  Print 
``\verb+Evil eradicates all trace of Good+'' if evil wins.  If there is a tie,
then print
``\verb+No victor on this battle field+''.

\vskip 16pt
\noindent
\setlength{\extrarowheight}{4pt}
\begin{tabularx}{\textwidth}{ | p{5cm} | X | }
\hline
\textbf{Sample Input} & \textbf{Sample Output} \\
\verbatiminput{GoodVersusEvil.sample.in}
&
\verbatiminput{GoodVersusEvil.sample.out}
\\
\hline
\end{tabularx}

\end{document}
